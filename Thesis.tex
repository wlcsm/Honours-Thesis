\documentclass{mimosis}

\usepackage{metalogo}
\usepackage{tikz-cd}

%%%%%%%%%%%%%%%%%%%%%%%%%%%%%%%%%%%%%%%%%%%%%%%%%%%%%%%%%%%%%%%%%%%%%%%%
% Some of my favourite personal adjustments
%%%%%%%%%%%%%%%%%%%%%%%%%%%%%%%%%%%%%%%%%%%%%%%%%%%%%%%%%%%%%%%%%%%%%%%%
%
% These are the adjustments that I consider necessary for typesetting
% a nice thesis. However, they are *not* included in the template, as
% I do not want to force you to use them.

% This ensures that I am able to typeset bold font in table while still aligning the numbers
% correctly.
\usepackage{etoolbox}

\usepackage[binary-units=true]{siunitx}
\DeclareSIUnit\px{px}

\sisetup{%
  detect-all           = true,
  detect-family        = true,
  detect-mode          = true,
  detect-shape         = true,
  detect-weight        = true,
  detect-inline-weight = math,
}

%%%%%%%%%%%%%%%%%%%%%%%%%%%%%%%%%%%%%%%%%%%%%%%%%%%%%%%%%%%%%%%%%%%%%%%%
% Hyperlinks & bookmarks
%%%%%%%%%%%%%%%%%%%%%%%%%%%%%%%%%%%%%%%%%%%%%%%%%%%%%%%%%%%%%%%%%%%%%%%%

\usepackage[%
  colorlinks = true,
  citecolor  = RoyalBlue,
  linkcolor  = RoyalBlue,
  urlcolor   = RoyalBlue,
  unicode,
  ]{hyperref}

\usepackage{bookmark}

%%%%%%%%%%%%%%%%%%%%%%%%%%%%%%%%%%%%%%%%%%%%%%%%%%%%%%%%%%%%%%%%%%%%%%%%
% Bibliography
%%%%%%%%%%%%%%%%%%%%%%%%%%%%%%%%%%%%%%%%%%%%%%%%%%%%%%%%%%%%%%%%%%%%%%%%
%
% I like the bibliography to be extremely plain, showing only a numeric
% identifier and citing everything in simple brackets. The first names,
% if present, will be initialized. DOIs and URLs will be preserved.

\usepackage[%
  autocite     = plain,
  backend      = bibtex,
  doi          = true,
  url          = true,
  giveninits   = true,
  hyperref     = true,
  maxbibnames  = 99,
  maxcitenames = 99,
  sortcites    = true,
  style        = numeric,
  ]{biblatex}

%%%%%%%%%%%%%%%%%%%%%%%%%%%%%%%%%%%%%%%%%%%%%%%%%%%%%%%%%%%%%%%%%%%%%%%%
% Some adjustments to make the bibliography more clean
%%%%%%%%%%%%%%%%%%%%%%%%%%%%%%%%%%%%%%%%%%%%%%%%%%%%%%%%%%%%%%%%%%%%%%%%
%
% The subsequent commands do the following:
%  - Removing the month field from the bibliography
%  - Fixing the Oxford commma
%  - Suppress the "in" for journal articles
%  - Remove the parentheses of the year in an article
%  - Delimit volume and issue of an article by a colon ":" instead of
%    a dot ""
%  - Use commas to separate the location of publishers from their name
%  - Remove the abbreviation for technical reports
%  - Display the label of bibliographic entries without brackets in the
%    bibliography
%  - Ensure that DOIs are followed by a non-breakable space
%  - Use hair spaces between initials of authors
%  - Make the font size of citations smaller
%  - Fixing ordinal numbers (1st, 2nd, 3rd, and so) on by using
%    superscripts

% Remove the month field from the bibliography. It does not serve a good
% purpose, I guess. And often, it cannot be used because the journals
% have some crazy issue policies.
\AtEveryBibitem{\clearfield{month}}
\AtEveryCitekey{\clearfield{month}}

% Fixing the Oxford comma. Not sure whether this is the proper solution.
% More information is available under [1] and [2].
%
% [1] http://tex.stackexchange.com/questions/97712/biblatex-apa-style-is-missing-a-comma-in-the-references-why
% [2] http://tex.stackexchange.com/questions/44048/use-et-al-in-biblatex-custom-style
%
\AtBeginBibliography{%
  \renewcommand*{\finalnamedelim}{%
    \ifthenelse{\value{listcount} > 2}{%
      \addcomma
      \addspace
      \bibstring{and}%
    }{%
      \addspace
      \bibstring{and}%
    }
  }
}

% Suppress "in" for journal articles. This is unnecessary in my opinion
% because the journal title is typeset in italics anyway.
\renewbibmacro{in:}{%
  \ifentrytype{article}
  {%
  }%
  % else
  {%
    \printtext{\bibstring{in}\intitlepunct}%
  }%
}

% Remove the parentheses for the year in an article. This removes a lot
% of undesired parentheses in the bibliography, thereby improving the
% readability. Moreover, it makes the look of the bibliography more
% consistent.
\renewbibmacro*{issue+date}{%
  \setunit{\addcomma\space}
    \iffieldundef{issue}
      {\usebibmacro{date}}
      {\printfield{issue}%
       \setunit*{\addspace}%
       \usebibmacro{date}}%
  \newunit}

% Delimit the volume and the number of an article by a colon instead of
% by a dot, which I consider to be more readable.
\renewbibmacro*{volume+number+eid}{%
  \printfield{volume}%
  \setunit*{\addcolon}%
  \printfield{number}%
  \setunit{\addcomma\space}%
  \printfield{eid}%
}

% Do not use a colon for the publisher location. Instead, connect
% publisher, location, and date via commas.
\renewbibmacro*{publisher+location+date}{%
  \printlist{publisher}%
  \setunit*{\addcomma\space}%
  \printlist{location}%
  \setunit*{\addcomma\space}%
  \usebibmacro{date}%
  \newunit%
}

% Ditto for other entry types.
\renewbibmacro*{organization+location+date}{%
  \printlist{location}%
  \setunit*{\addcomma\space}%
  \printlist{organization}%
  \setunit*{\addcomma\space}%
  \usebibmacro{date}%
  \newunit%
}

% Display the label of a bibliographic entry in bare style, without any
% brackets. I like this more than the default.
%
% Note that this is *really* the proper and official way of doing this.
\DeclareFieldFormat{labelnumberwidth}{#1\adddot}

% Ensure that DOIs are followed by a non-breakable space.
\DeclareFieldFormat{doi}{%
  \mkbibacro{DOI}\addcolon\addnbspace
    \ifhyperref
      {\href{http://dx.doi.org/#1}{\nolinkurl{#1}}}
      %
      {\nolinkurl{#1}}
}

% Use proper hair spaces between initials as suggested by Bringhurst and
% others.
\renewcommand*\bibinitdelim {\addnbthinspace}
\renewcommand*\bibnamedelima{\addnbthinspace}
\renewcommand*\bibnamedelimb{\addnbthinspace}
\renewcommand*\bibnamedelimi{\addnbthinspace}

% Make the font size of citations smaller. Depending on your selected
% font, you might not need this.
\renewcommand*{\citesetup}{%
  \biburlsetup
  \small
}

\DeclareLanguageMapping{english}{english-mimosis}

\addbibresource{Thesis.bib}

%%%%%%%%%%%%%%%%%%%%%%%%%%%%%%%%%%%%%%%%%%%%%%%%%%%%%%%%%%%%%%%%%%%%%%%%
% Fonts
%%%%%%%%%%%%%%%%%%%%%%%%%%%%%%%%%%%%%%%%%%%%%%%%%%%%%%%%%%%%%%%%%%%%%%%%

\ifxetexorluatex
  \setmainfont{Minion Pro}
\else
  \usepackage[lf]{ebgaramond}
  \usepackage[oldstyle,scale=0.7]{sourcecodepro}
  \singlespacing
\fi

\renewcommand{\th}{\textsuperscript{\textup{th}}\xspace}

\newacronym[description={Principal component analysis}]{PCA}{PCA}{principal component analysis}
\newacronym                                            {SNF}{SNF}{Smith normal form}
\newacronym[description={Topological data analysis}]   {TDA}{TDA}{topological data analysis}

\newglossaryentry{LaTeX}{%
  name        = {\LaTeX},
  description = {A document preparation system},
  sort        = {LaTeX},
}

\newglossaryentry{Real numbers}{%
  name        = {$\real$},
  description = {The set of real numbers},
  sort        = {Real numbers},
}

\makeindex
\makeglossaries

%%%%%%%%%%%%%%%%%%%%%%%%%%%%%%%%%%%%%%%%%%%%%%%%%%%%%%%%%%%%%%%%%%%%%%%%
% Incipit
%%%%%%%%%%%%%%%%%%%%%%%%%%%%%%%%%%%%%%%%%%%%%%%%%%%%%%%%%%%%%%%%%%%%%%%%

\title{\texttt{latex-mimosis}}
\subtitle{A minimal, modern \LaTeX{} package for typesetting your thesis}
\author{Bastian Rieck}

\begin{document}

\frontmatter
  \begin{titlepage}
  \vspace*{5cm}
  \makeatletter
  \begin{center}
    \begin{Huge}
      \@title
    \end{Huge}\\[0.1cm]
    %
    \begin{Large}
      \@subtitle
    \end{Large}\leavevmode\\
    %
    \emph{by}\\
    \@author
    %
    \vfill
    Mathematics Honours Thesis
    at\\
    \textsc{The Australian National University}
  \end{center}
  \makeatother
\end{titlepage}

\newpage

  \begin{center}
  \textsc{Abstract}
\end{center}
%
\noindent
%
<<<<<<< HEAD
A study and catalogue of the various polynomial multiplication algorithms with the focus on establishing practical crossover points and the development of a heuristic for selecting the most suitable one for the given situation.
=======
Scientific documents often use \LaTeX{} for typesetting. While numerous
packages and templates exist, it makes sense to create a new one. Just
because.
>>>>>>> 391a2f76657f563121cb864d112b1856505bd026


  \tableofcontents

\mainmatter

  %%%%%%%%%%%%%%%%%%%%%%%%%%%%%%%%%%%%%%%%%%%%%%%%%%%%%%%%%%%%%%%%%%%%%%%%
\chapter{Introduction}
%%%%%%%%%%%%%%%%%%%%%%%%%%%%%%%%%%%%%%%%%%%%%%%%%%%%%%%%%%%%%%%%%%%%%%%%

\begin{center}
  \begin{minipage}{0.5\textwidth}
    \begin{small}
      In which the reasons for creating this package are laid bare for the
      whole world to see and we encounter some usage guidelines.
    \end{small}
  \end{minipage}
  \vspace{0.5cm}
\end{center}

\noindent This package contains a minimal, modern template for writing your
thesis. While originally meant to be used for a Ph.\,D.\ thesis, you can
equally well use it for your honour thesis, bachelor thesis, and so
on---some adjustments may be necessary, though.

%%%%%%%%%%%%%%%%%%%%%%%%%%%%%%%%%%%%%%%%%%%%%%%%%%%%%%%%%%%%%%%%%%%%%%%%
\section{Why?}
%%%%%%%%%%%%%%%%%%%%%%%%%%%%%%%%%%%%%%%%%%%%%%%%%%%%%%%%%%%%%%%%%%%%%%%%

I was not satisfied with the available templates for \LaTeX{} and wanted
to heed the style advice given by people such as Robert
Bringhurst~\cite{Bringhurst12} or Edward R.\
Tufte~\cite{Tufte90,Tufte01}. While there \emph{are} some packages out
there that attempt to emulate these styles, I found them to be either
too bloated, too playful, or too constraining. This template attempts to
produce a beautiful look without having to resort to any sort of hacks.
I hope you like it.

%%%%%%%%%%%%%%%%%%%%%%%%%%%%%%%%%%%%%%%%%%%%%%%%%%%%%%%%%%%%%%%%%%%%%%%%
\section{How?}
%%%%%%%%%%%%%%%%%%%%%%%%%%%%%%%%%%%%%%%%%%%%%%%%%%%%%%%%%%%%%%%%%%%%%%%%

The package tries to be easy to use. If you are satisfied with the
default settings, just add
%
\begin{verbatim}
\documentclass{mimosis}
\end{verbatim}
%
at the beginning of your document. This is sufficient to use the class.
It is possible to build your document using either \LaTeX|, \XeLaTeX, or
\LuaLaTeX. I personally prefer one of the latter two because they make
it easier to select proper fonts.

%%%%%%%%%%%%%%%%%%%%%%%%%%%%%%%%%%%%%%%%%%%%%%%%%%%%%%%%%%%%%%%%%%%%%%%%
\section{Features}
%%%%%%%%%%%%%%%%%%%%%%%%%%%%%%%%%%%%%%%%%%%%%%%%%%%%%%%%%%%%%%%%%%%%%%%%

%%%%%%%%%%%%%%%%%%%%%%%%%%%%%%%%%%%%%%%%%%%%%%%%%%%%%%%%%%%%%%%%%%%%%%%%
\begin{table}
  \centering
  \begin{tabular}{ll}
    \toprule
    \textbf{Package}      & \textbf{Purpose}\\
    \midrule
      \texttt{amsmath}          & Basic mathematical typography\\
      \texttt{amsthm}           & Basic mathematical environments for proofs etc.\\
      \texttt{booktabs}         & Typographically light rules for tables\\
      \texttt{bookmarks}        & Bookmarks in the resulting PDF\\
      \texttt{dsfont}           & Double-stroke font for mathematical concepts\\
      \texttt{graphicx}         & Graphics\\
      \texttt{hyperref}         & Hyperlinks\\
      \texttt{multirow}         & Permits table content to span multiple rows or columns\\ 
      \texttt{paralist}         & Paragraph~(`in-line') lists and compact enumerations\\
      \texttt{scrlayer-scrpage} & Page headings\\
      \texttt{setspace}         & Line spacing\\
      \texttt{siunitx}          & Proper typesetting of units\\
      \texttt{subcaption} & Proper sub-captions for figures\\
    \bottomrule
  \end{tabular}
  \caption{%
    A list of the most relevant packages required~(and automatically imported) by this template.
  }
  \label{tab:Packages}
\end{table}
%%%%%%%%%%%%%%%%%%%%%%%%%%%%%%%%%%%%%%%%%%%%%%%%%%%%%%%%%%%%%%%%%%%%%%%%

The template automatically imports numerous convenience packages that
aid in your typesetting process. \autoref{tab:Packages} lists the
most important ones. Let's briefly discuss some examples below. Please
refer to the source code for more demonstrations.

%%%%%%%%%%%%%%%%%%%%%%%%%%%%%%%%%%%%%%%%%%%%%%%%%%%%%%%%%%%%%%%%%%%%%%%%
\subsection{Typesetting mathematics}
%%%%%%%%%%%%%%%%%%%%%%%%%%%%%%%%%%%%%%%%%%%%%%%%%%%%%%%%%%%%%%%%%%%%%%%%

This template uses \verb|amsmath| and \verb|amssymb|, which are the
de-facto standard for typesetting mathematics. Use numbered equations
using the \verb|equation| environment.
%
If you want to show multiple equations and align them, use the
\verb|align| environment:
%
\begin{align}
    V &:= \{ 1, 2, \dots \}\\
    E &:= \big\{ \left(u,v\right) \mid \dist\left(p_u, p_v\right) \leq \epsilon \big\}
\end{align}
%
Define new mathematical operators using \verb|\DeclareMathOperator|.
Some operators are already pre-defined by the template, such as the
distance between two objects. Please see the template for some examples. 
%
Moreover, this template contains a correct differential operator. Use \verb|\diff| to typeset the differential of integrals:
%
\begin{equation}
  f(u) := \int_{v \in \domain}\dist(u,v)\diff{v}
\end{equation}
%
You can see that, as a courtesy towards most mathematicians, this
template gives you the possibility to refer to the real numbers~$\real$
and the domain~$\domain$ of some function. Take a look at the source for
more examples. By the way, the template comes with spacing fixes for the
automated placement of brackets.

%%%%%%%%%%%%%%%%%%%%%%%%%%%%%%%%%%%%%%%%%%%%%%%%%%%%%%%%%%%%%%%%%%%%%%%%
\subsection{Typesetting text}
%%%%%%%%%%%%%%%%%%%%%%%%%%%%%%%%%%%%%%%%%%%%%%%%%%%%%%%%%%%%%%%%%%%%%%%%

Along with the standard environments, this template offers
\verb|paralist| for lists within paragraphs.
%
Here's a quick example: The American constitution speaks, among others, of
%
\begin{inparaenum}[(i)]
  \item life
  \item liberty
  \item the pursuit of happiness.
\end{inparaenum}
%
These should be added in equal measure to your own conduct. To typeset
units correctly, use the \verb|siunitx| package. For example, you might
want to restrict your daily intake of liberty to \SI{750}{\milli\gram}.

Likewise, as a small pet peeve of mine, I offer specific operators for \emph{ordinals}. Use \verb|\th| to typeset things like July~4\th correctly. Or, if you are referring to the 2\nd edition of a book, please use \verb|\nd|. Likewise, if you came in 3\rd in a marathon, use \verb|\rd|. This is my 1\st rule.

%%%%%%%%%%%%%%%%%%%%%%%%%%%%%%%%%%%%%%%%%%%%%%%%%%%%%%%%%%%%%%%%%%%%%%%%
\section{Changing things}
%%%%%%%%%%%%%%%%%%%%%%%%%%%%%%%%%%%%%%%%%%%%%%%%%%%%%%%%%%%%%%%%%%%%%%%%

Since this class heavily relies on the \verb|scrbook| class, you can use
\emph{their} styling commands in order to change the look of things. For
example, if you want to change the text in sections to \textbf{bold} you
can just use
%
\begin{verbatim}
  \setkomafont{sectioning}{\normalfont\bfseries}
\end{verbatim}
%
at the end of the document preamble---you don't have to modify the class
file for this. Please consult the source code for more information.

  \chapter{Evaluation and interpolation}\label{chp:eval-interp}

Here we will look at algorithms which use the Chinese Remainder Theorem to multiply polynomials as mentioned in Chapter \ref{chp:preliminaries}. The difficulty is in finding algorithms that can compute the isomorphism in the CRT and its inverse efficiently.

Here we find that it is most efficient to use the CRT over a set of linear divisors. That is, we consider multiplication in the ring $K[x] / f(x)$ where $f(x) = (x - \alpha_1)(x - \alpha_2) \ldots (x - \alpha_n)$ for some distinct $\alpha_1, \ldots, \alpha_n \in K$. So by the Chinese Remainder theorem we have
\[
    \frac{K[x]}{f(x)} \cong \frac{K[x]}{x - \alpha_1} \times \cdots \frac{K[x]}{x - \alpha_k}
\]
where $\alpha_1, \ldots, \alpha_k \in K$ constants. This tells us that if we have polynomials $a(x), b(x) \in K[x] / f(x)$. Then we could coerce $a(x)$ and $b(x)$ into $\frac{K[x]}{x - \alpha_1} \times \cdots \frac{K[x]}{x - \alpha_k}$ and combine them there in a way that corresponds to multiplication in $K[x] / f(x)$. 

Note that coercing $a(x)$ into $K[x] / (x - \alpha_i)$ is just $a(\alpha_i)$. Suppose $c(x) = a(x)b(x) \in K[x] / f(x)$ then we have $c(\alpha_i) = a(\alpha_i)b(\alpha_i)$. So to obtain $c(x)$ in the ring $K[x] / (x - \alpha_i)$ we simply need to multiply $a(\alpha_i)$ and $b(\alpha_i)$ together.

As we know, if $g(x) \in K[x]$ then $g(x) \mod (x - \alpha_i) = g(\alpha_i)$, this is the evaluation step. The step where we recover the original polynomial in $K[x]/f(x)$ is known as the interpolation step.

Evaluation at a single point can be performed in $\M{O}(n)$ time via Horner's Rule and is asymptotically optimal. Hence we could evaluate at $n + m$ distinct points in $\M{O}((n + m)^2)$ time. Lagrange interpolation can also be performed in $\M{O}((n+m)^2)$ time. Leading us to calculate this is $\M{O}((n + m)^2)$ time which is not better than the standard method we introduced before

Though this method appears that it would take longer than a direct calculation, we will show not only is there an asymptotically fast method of evaluating and interpolating polynomials at specifically chosen points, it is even quite practically efficient and is highly applicable with medium-sized inputs.

\section{Classical Algorithms reviewed as evaluation and interpolation}%
\label{sec:classical_algorithms_reviewed_as_evaluation_and_interpolation}

% This section comes from the Summary-of-Multiplication-Algorithms
We will now show how classical algorithms can be viewed as evaluation-interpolation procedures. Here we will present Karatsuba's algorithm and the Toom-Cook algorithms in the language of evaluation and interpolation.

Karatsuba's algorithm can be viewed as evaluation interpolation for $f(x) = x^2 - x$ (or $x^2 + x$ for a variation). However, we first need to use Kronecker substitution to put the polynomials into a form such that the degree is less than two. To do this, we introduce the variable $y = x^{n/2}$ where $n$ is the degree of the polynomials. So we obtain
\[
    a(x)(y) = a_1(x)y + a_0(x), \qquad b(x)(y) = b_1(x)y + b_0(x)
\]
We then apply the Chinese remainder theorem. Evaluate at $y = 0$ and $y = 1$, to get
\begin{align*}
    a(x)(0) &= a_0(x), \qquad a(x)(1) = a_1(x) + a_0(x)\\
    b(x)(0) &= b_0(x), \qquad b(x)(1) = b_1(x) + b_0(x)
\end{align*}

Apply the Chinese remainder theorem to get $c(x) = a_0b_0 + ((a_0 + a_1)(b_0 + b_1) - a_0b_0)x \in K[x]/(x^2 - x)$. 

Then we ``evaluate at infinity'' to get the $x^2$ term. This could be interpreted as substituting $x^{-1}$ for $x$, multiplying by $x$ and then quotienting by $x$. The idea is that $a(x)$ evaluated at infinity is just the coefficient of the largest term. 

I think I found a better interpretation of this. The polynomial $a(x)b(x)$ may have a non-zero $x^2$ term, but we have used the Chinese remainder theorem to put it into $K[x] / (x^2 - x)$. Hence we have to undo a modular wrapping. That is, there is a unique $c(x) = c_0 + c_1x + c_2x^2$ such that $c(x) \cong a_0b_0 + ((a_0 + a_1)(b_0 + b_1) - a_0b_0)x \mod x^2 - x$. Namely $c(x) \cong c_0 + (c_1 + c_2)x \mod x^2 - x$. We know for sure that $c_2 = a_1b_1$ (i.e. the evaluation at infinity), therefore we can conclude that $a_0b_0 + ((a_0 + a_1)(b_0 + b_1) - a_0b_0)x$ corresponds to the polynomial $a_0b_0 + ((a_0 + a_1)(b_0 + b_1) - a_0b_0 - a_1b_1)x + a_1b_1x^2 \in K[x]$

The Toom-Cook algorithm generalises Karatsuba's method and evaluates at several points $-k + 1, \ldots, k - 1$ and is recovered similarly, although there are many variants with the Toom-Cook algorithm depending on how the result is interpolated. It is asymptotically superior to Karatsuba's method, but practically overtakes Karatsuba's method at only a large input size.

\section{The Fast Fourier Transform}

The Discrete Fourier Transform of a function is defined by evaluating the function at the roots of unity. The Fast Fourier Transform is an algorithm for evaluating the DFT (and its inverse) in $\M{O}(n \log n)$ time.

\subsection{The Discrete Fourier Transform}

Let $R$ be a commutative ring. The Discrete Fourier Transform is defined by evaluating a function at roots of unity.

% This definition obtained from the Algebraic Complexity theory book
\begin{definition}[Roots of Unity]
  Let $N$ be an integer, then an element $\alpha \in R$ is a \emph{principal $N^{\tx{th}}$ root of unity} if
  \begin{enumerate}
    \item $\alpha^N = 1$
    \item $\alpha^p - 1$ is not a zero divisor for all $1 \leq p < N$.
  \end{enumerate}
\end{definition}


\begin{definition}[Discrete Fourier Transform]
    The $k^{\tx{th}}$ Fourier coefficient of the Discrete Fourier Transform of samples $x_0, \ldots, x_{N-1}$ with roots of unity $(\omega_N^i)_{i=0}^{N-1}$ is
\[
    X_k = \sum^{N-1}_{i=0}x_i\omega_{N}^{ik}
\]
for $0 \leq k \leq N-1$.
\end{definition}

Note that when computed directly, each of the $N$ Fourier coefficients take $\M{O}(N)$ time to compute. Hence computing all coefficients takes $\M{O}(n^2)$ time.


% <><><><><><><><><> FFT <><><><><><><><><>%
\section{The Fast Fourier Transform}

The Fast Fourier Transform is an algorithm for calculating the DFT with $N$ samples in $\M{O}(N \log N)$ time. First developed by Gauss in 1805 \cite{gauss}, this algorithm has had a profound impact on the course of human computing, making many options possible that were not considered before.\\
There are many different variations of the Fast Fourier Transform, but here we will analyse the original algorithm described in the landmark Cooley-Tukey paper \cite{10.2307/2003354}.

\begin{theorem}[Fast Fourier Transform]\label{thm:fft}
    If $N$ is a power of two, then the DFT with $N$ samples can be calculated in $\M{O}(N\log N)$ time.
\end{theorem}

Here we use a standard divide-and-conquer approach whereby the DFT is broken up into two smaller DFTs each with $N/2$ elements which can then be computed recursively and then combined quickly. Since we already established that the DFT can be computed in $\M{O}(N^2)$, the two DFTs of size $N/2$ can be computed in approximately $(N/2)^2 = N^2/4$ times, thus computing both of them takes $N^2 / 2$. Recombining the two takes $\M{O}(N)$ time. Therefore for large $N$ we would expect that $N^2/2 + \M{O}(N) < N$. This procedure repeats recursively until only one element (or some predefined constant number) remains in each DFT, so we would expect this to be much faster than the direct $\M{O}(N^2)$ calculation.

We will now prove a lemma which formalises this intuition. The lemma serves to outline the general algorithm for computing the DFT efficiently, and the remainder of the proof verifies the time complexity of the algorithm.

\begin{lemma}
    Let $C(N)$ be the number of calculations required to compute the DFT with $N = 2^n$ elements. Then we have
    \begin{equation}
        C(N) = 2 C(N/2) + pN \label{eq:fftlem}
    \end{equation}
    for an absolute constant $p$.
\end{lemma}

\begin{proof}
From the definition of the DFT we have
\[
    X_k = \sum^{N-1}_{i=0}x_i\omega^{ik}
\]
Rearranging the expression we obtain
\begin{align}
    X_k
    &= \sum^{N-1}_{i=0}x_i\omega^{ik} \nonumber\\
    &= \sum^{N-1}_{i=0}x_{2i}\omega^{2ik} \;+\; \sum^{N-1}_{i=0}x_{2i+1} \omega^{(2i+1)k} \nonumber\\
    &= \sum^{N/2-1}_{i=0}x_{2i}\omega^{2ik} \;+\; \omega^k \sum^{N/2-1}_{i=0}x_{2i+1}\omega^{2ik} \label{eq:keystep}
\end{align}
Now note that $\omega^2$ is a root of unity of order $N/2$. Hence the two terms in the last line are DFTs of length $N/2$.

TODO This explanation is not very good
A very important observation to make is that the DFT has an $N$-dimensional image $[X_1, \ldots, X_k]$.

So in the term $\sum^{N/2-1}_{n=0}x_{2n}\omega^{nk}_{N/2}$, $k$ should range from $0$ to $N$, but it is only necessary to calculate it for $0 \leq k < N/2$. This means that we need to evaluate the two sums on the RHS only for $0 \leq k \leq N/2$.

Therefore we have transformed a DFT with $N$ elements, into two DFTs of $N/2$ elements, one containing all the samples with odd index, and one containing all the samples at an even index.\\
Computing each of the two smaller DFTs takes $2C(N/2)$. Then multiplying by $\omega^k_N$ and adding the two DFTs together for all $0 \leq k < N$ will take $\M{O}(N)$ time. \\
So we have
\[
    C(N) = 2 C(N/2) + pN
\]
for some constant $p$.\\
This concludes the proof of the lemma.
\end{proof}

\begin{proof}[Proof of Theorem \ref{thm:fft}]
    Notice that since we are halving the size of the DFT at every recursive call, the maximum recursion depth is $\log_2 N$ (how many times the function can call itself until it reaches the base case).
    By expanding these recursive calls we obtain
    \begin{align*}
    C(N) &= 2\,(\;\cdots\; (2C(1) + p(N/2^{n-1})) + pN/2^{n-2}) + \cdots ) + pN\\
         &= 2^{n}C(1) + pN/2^{(n-k)}\sum^n_{k=1} 2^{n-k}\\
         &= 2^{\log_2(N)}C(1) + pN\log_2(N)\\
         &= NC(1) + pN\log_2(N)
    \end{align*}
    Thus $C(N) = \M{O}(N \log N)$.
\end{proof}

\begin{remark}
    To perform the FFT, we need to pad the polynomials with zeros until the resulting polynomial has length $n + m$, and then rounded up to a power of two. This means that it can cause bloat even for dense polynomials, up to a factor of two. If we know our polynomials to be sparse, then this is even worse, and that is why it is terrible in the case for multivariate polynomials since almost all multivariate polynomials of interest are sparse.

    For instance, let $a = x^{129} + 1$ and $b = x^{128} + 1$, then we have $n + m = 257$, and so we have to round it up to the next power of two $512$. Hence, despite this being trivial to compute using the school-book method, it ends up taking this algorithm a really long time.

    There are ways to mitigate this, but ultimately the FFT algorithm is not well suited for sparse polynomials.

    Another useful technique we can use if a polynomial's degree has another prime factor (other than two) is that we can just leave that prime factor as the base case which we can then handle through the naive algorithm.
\end{remark}

\subsection{Mixed-radix FFT}

To get around the fact that you need to round up to a power of two, one could perform the DFT with an arbitrary radix. In our previous formulation, we split the DFT into two smaller DFTs each time. Here $2$ is the \emph{radix} as that is our splitting factor. We could generalise the formula for an arbitrary radix.

Say we want to do the radix-$m$ DFT:
Performing the same steps we obtain
\[
    F_n(k) = \sum^{n-1}_{i=0} a_0\omega_n^{ik}
\]
Then instead of breaking it into $2$ pieces, break it into $m$ pieces
\begin{align*}
    F_n(k) &= \sum^{\frac{n}{m}-1}_{i=0} x_{mi}\omega_n^{mik} + \sum^{\frac{n}{m}-1}_{i=0} x_{mi+1}\omega_n^{(mi+1)k} + \ldots + \sum^{\frac{n}{m}-1}_{i=0} x_{mi+2}\omega_n^{(mi+2)k}\\
          &= \sum^{\frac{n}{m}-1}_{i=0} x_{mi}\omega_n^{mik} + \omega_n^k\sum^{\frac{n}{m}-1}_{i=0} a_{mi+1}\omega_n^{mik} + \ldots +  \omega_n^{2k}\sum^{\frac{n}{m}-1}_{i=0} x_{mi+2}\omega_n^{mik}\\
          &= F^0(k) + \omega_{\frac{n}{m}}^k F^1(k) + \ldots + \omega_{\frac{n}{m}}^{(m-1)k} F^{m-1}(k).
\end{align*}

But also notice that each of the smaller DFTs $F^i(k)$ have $\frac{n}{m}jk$ elements, in other words if $k = p\frac{n}{m} + q$ we have that $F^i(k) = F^i(q)$ so we further reconstruct it as follows
\begin{align*}
    F_n\bb{\frac{n}{m}p + q} &= \sum^{m-1}_{i=0} F^i\bb{\frac{n}{m} + q}\omega^{i(\frac{n}{m}p + q)}_n\\
    &= \sum^{m-1}_{i=0} \omega^{ip\frac{n}{m}}_n \omega_n^{iq} F^i(q)\omega^{ip\frac{n}{m}}_n\\
    &= \sum^{m-1}_{i=0} (\omega^{iq}_n F^i(k))\omega^{ip}_m.
\end{align*}
Hence if we fix $q$, this is now a DFT of $m$ elements. Thus we can calculate $F(\frac{n}{m}j + k)$ for all $0 \leq j < m$ in $F_m$ time.

Calculating all the sub-DFTs takes $F_{\frac{n}{m}}$ time since there are $m$ of size $\frac{n}{m}$. Combining them together is equivalent to calculating $\frac{n}{m}$ sub-DFTs of size $m$, which contributes $\frac{n}{m}F_{m}$ time.\\
In total this is
\[
    mF_{\frac{n}{m}} + \frac{n}{m}F_m
\]
time.

If we set $m = \sqrt{n}$ then we would get $2\sqrt{n}F_{\sqrt{n}}$. TODO Double check this bound.

However, in general, it is always faster to use the smallest radix possible; two being the most desirable since in practice computers perform all operations in base two and so there are many optimisations. One such optimisation is the \emph{reverse bit encoding}, where we can organise $X_1, \ldots, X_k$ in memory so that the entire FFT algorithm can be performed in-place on the data (without having to copy elements). This is done by putting $X_i$ at index $\tx{rev}(i, n)$, which is the function which reverses the order of the $n$-bit representation of $i$.

\subsection{Sch\"{o}nage and Strassen Integer Multiplication Algorithm}
\label{subsec:schon-strass}

The problem with multiplying polynomials in $\Z$ is that $\Z$ does not have roots of unity. One can consider $\Z[x]$ as a subring of $\C[x]$ and then apply the FFT algorithms, rounding to the nearest integer to convert the result back into $\Z$. That is a popular approach for medium-sized inputs; however, often when we work in $\Z[x]$ we are interested in exact solutions whereas when we work in $\C$ we accept that there will be some error. If the error becomes too large then rounding to the nearest integer will yield the incorrect result, and so one must be able to guarantee the error is suitably controlled. Indeed this is how \cite{nlogn} achieves $\M{O}(n\log n)$ complexity. Another option is to use the CRT to put the result into finite fields, but we will cover than later. For now, we will look at the most popular algorithm for large $\Z[x]$ multiplication, Sch\"{o}nage and Strassen's second integer multiplication. The algorithm's original intention was for integer multiplication, but it ends up converting integers into polynomials in $\Z[x]$ first using Kronecker substitution from Chapter \ref{chp:preliminaries}.

The Sch\"{o}nage and Strassen Integer multiplication algorithm works by reorganising the polynomial into another form where the coefficients belong in a ring that does have roots of unity, so-called ``synthetic roots''.


Observe that in the ring $\Z[X] / (X^n + 1)$ the polynomial $X$ is a $2n^{\tx{th}}$ root of unity. So we now want to convert our original polynomial into one with coefficients in a ring of the above form. To do this we can use Kronecker substitution to partition the polynomial into $n^{\frac{1}{2}}$ pieces with the substitution $y = x^{n^{\frac{1}{2}}}$. We can then apply the FFT algorithm: The FFT step takes $\M{O}(n^{\frac{1}{2}}\log n^{\frac{1}{2}} \times n^{\frac{1}{2}}) = \M{O}(n \log n)$ where the arithmetic on the $n^{\frac{1}{2}}$ blocks is $\M{O}(n^{\frac{1}{2}})$ time and the main algorithm is $\M{O}(n^{\frac{1}{2}}\log n^{\frac{1}{2}})$. To perform the elementwise multiplications we recursively call the algorithm again. This gives us the recursive form
\[
    C(n) < n^{\frac{1}{2}}C(n^{\frac{1}{2}}) + \M{O}(n \log n)
\]
This can be solved to obtain $\M{O}(n \log n \log \log n)$. 
% TODO I have not taken into account the fact that n might not be square and we also need to round n to the largest power of two.

% \section{Rader's Algorithm}%
% \label{sec:Rader's Algorithm}

% In this one we find an efficient algorithm if the transform length $n$ is prime and $n - 1$ has small coefficients.

% This algorithm is often cited in many papers but people also say it ends up being slower than the FFT algorithm so idk. Also, Winograd generalised it to powers of primes. In  they say that they use Rader's algorithm and it produces a speed-up of approximately 2x, but they also say that they might have modified the algorithm a bit

% I think it might be more beneficial in the case of the finite field because its much is harder to use a DFT there. (And I have put a little section there about it).


% This ensures that the subsequent sections are being included as root
% items in the bookmark structure of your PDF reader.
\bookmarksetup{startatroot}
\backmatter

  \begingroup
    \let\clearpage\relax
    \glsaddall
    \printglossary[type=\acronymtype]
    \newpage
    \printglossary
  \endgroup

  \printindex
  \printbibliography

\end{document}
