\chapter{Evaluation and interpolation}\label{chp:eval-interp}

The algorithms for polynomial multiplication that we have seen so far have been direct evaluations of the multiplication formula we introduced in Chapter \ref{chp:preliminaries}. Soon after the rediscovery of the Fast Fourier Transform in 1965, the \textit{evaluation-interpolation strategy} became a more popular choice. Rather than evaluating the multiplication formula directly, this strategy involves sampling the input polynomials at several points and interpolating the result from the product of the samples. Though this would appear to take more time that than the direct approach, it is remarkably efficient in practice and allows us to use techniques from Fourier analysis to multiply polynomials in $\M{O}(n \log n)$ time in the RAM model. In fact, we will revisit Karatsuba's algorithm from the previous chapter and show how it too can be interpreted as using the evaluation-interpolation strategy.

We will present an algebraic formulation of the Fast Fourier transform and visit Sch\"{o}nage and Strassen's integer multiplication algorithm generalised for polynomials which is currently used as part of the GMP Multiple Precision Arithmetic library for integers with around $16000$ bits \cite{gmp}.

% TODO polish
For an intuitive explanation of how it works, consider the multiplication $h = fg \in K[x] / f(x)$. It follows that $h(\alpha) = f(\alpha)g(\alpha)$ for all $\alpha \in K$. Thus by evaluating the polynomials $f$ and $g$ at $\deg f + \deg g + 1$ distinct points, we can multiply the samples together and interpolate $h$. We can correctly recover $h$ since $\deg h \le \deg f + \deg g$ and any polynomial of degree $n$ can be interpolated from $n + 1$ distinct samples.

More formally, let $n = \deg f + \deg g$ and $f, g \in K[x]/(x^n - 1)$. We can consider the evaluation step as the map from $K[x]/(x^n - 1) \to K[x]/(x - \alpha_1) \times \cdots \times K[x]/(x - \alpha_n)$, and the interpolation step is its inverse. The guarantees of correctness comes from the Chinese Remainder Theorem which asserts the two rings are isomorphic.

The process of coercing $a$ and $b$ into $K[x]/ (x - \alpha_i)$ is called the \emph{evaluation} step. The step where we recover the original polynomial in $K[x]/f(x)$ is called the \emph{interpolation} step.

\medskip

Evaluation at a single point can be performed in $\M{O}(n)$ time via Horner's Rule and is asymptotically optimal since we must perform some operation with each term in the polynomial at least once. However, applying this procedure to evaluate the polynomial at $n$ distinct points is therefore $\M{O}(n^2)$ time. Similarly, we may use Lagrange interpolation to interpolate a degree $n$ polynomial in $\M{O}(n^2)$ time. Neither of these methods provide any improvement over the standard schoolbook, so this chapter will present an asymptotically faster algorithms for evaluation and interpolation.

\section{Karatsuba's Algorithm as Evaluation-Interpolation}%
\label{sec:Karatsuba's Algorithms as Evaluation-Interpolation}

% This section comes from the Summary-of-Multiplication-Algorithms
Karatsuba's algorithm can be viewed as a method for evaluating the Chinese Remainder Theorem for the isomorphism
\[
    \frac{K[x]}{x^2 - x} \cong \frac{K[x]}{x} \times \frac{K[x]}{x-1},
\]
along with an \emph{evaluation at infinity} which we will explain shortly\footnote{There are variations that use $x^2 + x$ as the divisor which may be more computationally efficient in certain circumstances. See Knuth, Art of Computer Programming Vol 2. \cite{knuthv2}}.\\
However, in order to use this isomorphism we first need to use Kronecker substitution to express the input polynomials as elements of $K[x][y]/(y^2 - y)$. To do this, we introduce the variable $y = x^{\floor{n/2}}$ where $n$ is the degree of the polynomials
\[
    f(x)(y) = f_1(x)y + f_0(x), \qquad g(x)(y) = g_1(x)y + g_0(x).
\]
Evaluating this at $y = 0$ and $y = 1$ gives
\begin{align*}
    f(x)(0) &= f_0(x), \qquad f(x)(1) = f_1(x) + f_0(x)\\
    g(x)(0) &= g_0(x), \qquad g(x)(1) = g_1(x) + g_0(x)
\end{align*}

We can then apply the applying the Chinese remainder theorem to evaluate the isomorphism
\begin{equation}\label{eq:karatsuba-crt}
    h(x) = f_0g_0 + ((f_0 + f_1)(g_0 + g_1) - f_0g_0)x \in K[x]/(x^2 - x).
\end{equation}
The problem now is that our result is in $K[x][y]/(y^2 - y)$, however, the true result of the multiplication of $f(x)(y)$ and $g(x)(y)$ could have a $y^2$ term. Therefore we must recover this term by \emph{evaluating at infinity}. This amounts to using the largest term of $f(x)(y)$ and $g(x)(y)$ to interpret the result.

\medskip 

Given that $h(y) = f(y)g(y) = (f_0 + f_1y)(g_0 + g_1 y)$, we know the greatest term in the result could be $y^2$, and that the coefficient of this term would be $f_1g_1$. Therefore if we were to coerce $h(y)$ into $K[y]/(x^2 - x)$ we would obtain $f_0g_0 + ((f_0 + f_1)(g_0 + g_1) + f_1g_1)x$. Thus to undo this operation in our result \eqref{eq:karatsuba-crt} we obtain $f_0g_0 + ((f_0 + f_1)(g_0 + g_1) - f_0g_0 - f_1g_1)x + f_1g_1x^2 \in K[x]$ which matches our original formulation of Karatsuba's algorithm.

\section{The Fast Fourier Transform}

% TODO find out who originally posed this, I think maybe it was Stockham

The Discrete Fourier Transform (DFT) is an powerful mathematical technique for performing Fourier analysis in a wide number of applications. Most notably, signal processing when it can be used to transform data between sample space and the frequency domain. The Fast Fourier Transform (FFT) is an algorithm for evaluating the DFT (and its inverse) in $\M{O}(n \log n)$ time. The na\"{i}ve algorithm for evaluating the DFT takes $\M{O}(n^2)$ time, which rendered it infeasible for most practical applications. So when Cooley and Tukey released their landmark paper on FFT in 1965, it spawned flurry of development across many fields. One example is the development of FFT based algorithms for polynomial multiplication, which was the first $\M{O}(n \log n)$ algorithm for the problem.


(TODO make better)
The DFT is the evaluation of a function at \textit{roots of unity}, and its inverse is a means of recovering the original function from samples. Hence this can be seen as an instance of the evaluation-interpolation strategy. 
In this section we formulate both the DFT and FFT and explore the relationship between the DFT and polynomial multiplication. In the end showing them to be equivalent problems through Bluestein's Chip transform.

\subsection{The Discrete Fourier Transform}

Let $R$ be a commutative ring. We are familiar with the complex roots of unity given as the solutions to the equation $x^n - 1$ for $n \in \N$. However many rings admit elements that behave similarly and can be used in the FFT algorithm. 

% This definition obtained from the Algebraic Complexity theory book
\begin{definition}[Roots of Unity]
  Let $N$ be an integer, then an element $\alpha \in R$ is a \emph{principal $N^{\tx{th}}$ root of unity} if
  \begin{enumerate}
    \item $\alpha^N = 1$
    \item $\alpha^p - 1$ is not a zero divisor for all $1 \leq p < N$.
  \end{enumerate}
\end{definition}

It is clear that the complex roots of the equation $x^N - 1$ satisfy the above definition. Though there are many other coefficient algebras that also have roots of unity for a particular $N$. Chapter \ref{chp:integer-rings} will establish results for the existence of roots of unity in integer rings as well as methods for finding them. Another popular choice is the ring $K[x]/(x^n + 1)$ which has $x$ as a $2n^{\tx{th}}$ root of unity (also called a \textit{synthetic root of unity}). We will revisit this ring in Section \ref{sec:schon-strass} when discussing Sch\"{o}nage and Strassen's integer multiplication scheme. 

\begin{definition}[Discrete Fourier Transform]
    The $k^{\tx{th}}$ Fourier coefficient of the Discrete Fourier Transform of samples $x_0, \ldots, x_{N-1}$ with roots of unity $(\omega_N^i)_{i=0}^{N-1}$ is given as
    \[
        X_k = \sum^{N-1}_{i=0}x_i\omega_{N}^{-ik}
    \]
    for $0 \leq k \leq N-1$.

    The transform has a well defined inverse given as 
    \[
        x_k = \frac{1}{N}\sum^{N-1}_{i=0}x_i\omega_{N}^{ik}
    \]
\end{definition}

Note that when computed directly, each of the $N$ sums in the forward or inverse DFT take $\M{O}(N)$ time to compute. Hence computing one transform takes $\M{O}(N^2)$ time, which is still no better than the schoolbook method.

\subsection{The Fast Fourier Transform Algorithm}

The Fast Fourier Transform (FFT) is an algorithm for calculating the DFT with $N$ samples in $\M{O}(N \log N)$ time. First developed by Gauss in 1805 \cite{gauss}, this algorithm has had a profound impact on the course of human computing. Due to its efficiency on modern computer hardware, it was able to make many previously incomputable problems in signal processing tractable.
There are many variations of the Fast Fourier Transform, but here we will analyse the original algorithm described in the landmark Cooley-Tukey paper \cite{fft}.

\begin{theorem}[Fast Fourier Transform]\label{thm:fft}
    If $N$ is a power of two, then the DFT with $N$ samples can be calculated in $\M{O}(N\log N)$ time.
\end{theorem}

The FFT uses a standard divide-and-conquer approach whereby the DFT is split into two smaller DFTs each with $N/2$ elements and computed recursively. Since we already established the DFT can be computed in $\M{O}(N^2)$ time, the two sub-DFTs of size $N/2$ can be computed in approximately $(N/2)^2 = N^2/4$ time (ignoring any constants). Thus computing both sub-DFTs takes $N^2 / 2$ time.\\
Recombining two DFTs takes $\M{O}(N)$ time. Therefore in total the cost is approximately $N^2/2 + \M{O}(N)$. For large enough $N$, we have essentially halved the time it takes to compute the entire DFT. By recursively subdividing the DFT into progressively smaller DFTs we would expect to obtain a complexity similar to $\M{C}(n) = nC(1) + \sum^n_{i=1} \M{O}(N)$ where the $n\M{C}(1)$ term comes from performing the DFT on an input of size $1$ (in which case the DFT is the identity so $C(1) = 0$), and the $\sum^{\log_2 n}_{i=1} \M{O}(N)$ comes from recombining all the sub-DFTs. In total we have $\M{C}(n) = \M{O}(N \log N)$.

The following lemma which formalises this intuition and the general algorithm for computing the DFT efficiently. The remainder of the proof verifies the time complexity.

\begin{lemma}
    Let $C(N)$ be the number of calculations required to compute the DFT with $N = 2^n$ elements. Then
    \begin{equation}
        C(N) \le 2 C(N/2) + pN \label{eq:fftlem}
    \end{equation}
    for a positive constant $p > 0$.
\end{lemma}

\begin{proof}
From the definition of the DFT, for each $0 \le k \le N-1$ we have
\[
    X_k = \sum^{N-1}_{i=0}x_i\omega_N^{ik}.
\]
Throughout the rest of this section we will write $X_k$ with the understanding that $X_{k} = X_{k \mod N}$.

The crucial step in the FFT, is to split the DFT into a sum of $x$ terms with even exponents and one with odd exponents
\begin{align}
    X_k
    &= \sum^{N-1}_{i=0}x_i\omega_N^{ik} \nonumber\\
    &= \sum^{N-1}_{i=0}x_{2i}\omega_N^{2ik} \;+\; \sum^{N-1}_{i=0}x_{2i+1} \omega_N^{(2i+1)k} \nonumber\\
    &= \sum^{N/2-1}_{i=0}x_{2i}\omega_N^{2ik} \;+\; \omega_N^k \sum^{N/2-1}_{i=0}x_{2i+1}\omega_N^{2ik}\\
    &= \sum^{N/2-1}_{i=0}x_{2i}\omega_{N/2}^{ik} \;+\; \omega_N^k \sum^{N/2-1}_{i=0}x_{2i+1}\omega_{N/2}^{ik} \label{eq:keystep}
\end{align}
where the last line follows from the fact that $\omega_N^2 = \omega_{N/2}$. Hence the two terms in the last line are DFTs of length $N/2$. Moreover we have also obtained an expression for how to recombine these two sub-DFTs.


Notice that
\begin{align*}
    X_{k + N/2}
    &= \sum^{N/2-1}_{i=0}x_{2i}\omega_{N/2}^{i(k + N/2)} \;+\; \omega_N^{k+ N/2} \sum^{N/2-1}_{i=0}x_{2i+1}\omega_{N/2}^{i(k + N/2)}\\
    &= \sum^{N/2-1}_{i=0}x_{2i}\omega_{N/2}^{ik} \;-\; \omega_N^k \sum^{N/2-1}_{i=0}x_{2i+1}\omega_{N/2}^{ik},
\end{align*}
since $\omega_{N/2}^{N/2} = 1$ and $\omega_N^{N/2} = -1$.\\
So the sub-DFTs in $X_k$ are the same as the sub-DFTs in $X_{k + N/2}$, the only difference is that they are recombined in a different way.

Computing each of the two smaller DFTs takes $2C(N/2)$. Then multiplying by $\omega^k_N$ and adding the two DFTs together for all $0 \leq k < N$ will take $\M{O}(N)$ time. \\
So we have
\[
    C(N) \le 2 C(N/2) + pN
\]
for some constant $p > 0$.
\end{proof}

\begin{proof}[Proof of Theorem \ref{thm:fft}]
    We will show by induction that $\M{C}(N) \le pN\log_2 N$ for all $N$ a power of two.\\
    We know this is true for $N = 1$, because in that case the DFT is simply the identity so $\M{C}(1) = 0 \le p1 \log_2 1$.
    Suppose $\M{C}(M) \le pM\log_2 M$ holds for all $M$ a power of two less than $N$. Then
    \begin{align*}
        C(N)
        &= 2\M{C}(N/2) + pN\\
        &= pN\log_2 (N/2) + pN\\
        &= pN\log_2 N - pN\log_2 2 + pN\\
        &= pN\log_2 N
    \end{align*}
    Therefore by the principle of mathematical induction, we conclude that $C(N) = \M{O}(N \log N)$ for all $N$ a power of two.
\end{proof}


\subsection{Mixed-radix FFT}

In our previous formulation, we split the DFT into two smaller DFTs each time. Here $2$ is the \emph{radix} as that is our splitting factor. The FFT can be generalised to an arbitrary radix. In practice, this does little to improve the performance in most situations, but it is necessary in rings where $2$ is not invertible

Say we want to do the radix-$m$ DFT where $m$ divides $n$:\\
Performing the same steps we obtain
\[
    F_n(k) = \sum^{n-1}_{i=0} a_0\omega_n^{ik}.
\]
Instead of breaking $F_n(k)$ into two pieces, break it into $m$ pieces,
% TODO make sure this is correct. 
\begin{align*}
    F_n(p_1\frac{n}{m} + p_2) &= \sum^{m-1}_{j=0} \sum^{\frac{n}{m}-1}_{i=0} x_{mi + j}\omega_n^{(mi + j)(p_1 \frac{n}{m} + p_2)}\\ 
                              &= \sum^{m-1}_{j=0} \bb{\sum^{\frac{n}{m}-1}_{i=0} x_{mi + j}\omega_n^{mi p_2}} \omega_n^{j(p_1\frac{n}{m} + p_2)}\\
                              &= \sum^{m-1}_{j=0} \bb{\omega_n^{jp_2}\sum^{\frac{n}{m}-1}_{i=0} x_{mi + j}\omega_n^{mi p_2}} (\omega_n^{\frac{n}{m}})^{p_1 j}
\end{align*}

% TODO say what a twiddle factor is

Just as the original FFT first splits it into two sub-DFTs, one with all the coefficients at odd indices and the other the even indices. Each of the inner sums in the above expression correspond to a sub-DFT, and the $i^{\text{th}}$ one consist of all coefficients $a_r$ where $r = i \mod m$. The reason we split up the $\omega_n^{j(\frac{n}{m}p_1 + p_2)}$ in the last line is so that we can combine all of these sub-DFTs using a DFT of size $\frac{n}{m}$, we simply need to multiply each of the inner DFTs by $\omega_n^{jp_2}$.

Calculating all the sub-DFTs takes $F_{\frac{n}{m}}$ time since there are $m$ of size $\frac{n}{m}$. Multiplying the inner DFTs by a twiddle factor (plus computing the twiddle factors) is $O(n)$. Combining them together is equivalent to calculating $\frac{n}{m}$ sub-DFTs of size $m$, which contributes $\frac{n}{m}F_{m}$ time.\\
In total this is
\[
    mF_{\frac{n}{m}} + \frac{n}{m}F_m + O(n)
\]
time.

Note that in a Turing machine, the $m$ inner sub-DFTs also require us to reorganise $x$ into $n_1$ consecutive vectors of size $\frac{n}{m}$ and vice versa. This can be achieved in $O(n \log( \min \{m, \frac{n}{m}\}\tx{bit}(R))$ time where $\tx{bit}(R)$ is the number of bits required to store an elements of $R$ \cite{ffnlogn}. We will revisit this fact in Section \ref{chp:asymptotic} where we will perform our computations in the Turing machine.

From the above recursive formula we can see that changing $m$ does not change the asymptotic complexity; indeed, since we consider $m$ to be fixed we have $F_m$ is a constant, and so asymptotically this is equivalent to the recursive formula $mF_{\frac{n}{m}} + \M{O}(n)$ whose solution is $\M{O}(n \log n)$ as before. 

\medskip
(TODO polish this)
\medskip

So the main difference is a constant factor. Note however that it is much easier to structure a FFT with radix $2$ in memory since each recursive step requires us to combine two elements, so combining $m$ requires more jumping around in the memory. But Primarily because a computer does all computation in base two so it is much easier overall.

For example when using a radix of 2, we can structure the elements in their \emph{reverse bit encoding} to perform all operations in place. The reverse bit encoding is often just one CPU instruction in modern CPUs but it is much more complicated to reorganise the elements if the radix is arbitrary.

\begin{remark}
    (TODO Reword this)

    \medskip

    Our formulation of the FFT is only valid for powers of two, so we may need to pad our inputs with zeros to satisfy this requirement. In order to avoid this padding as much as possible, we may consider constructing a mixed-radix FFTs that generalises the FFT to transform sizes that are not powers of two. However, we note that these tend to be much more difficult to implement efficiently and in typical cases, it is more efficient to pad the polynomials in order to use the standard FFT than it is to use a mixed-radix DFT. This is because modern computers can perform many operations much more efficiently when everything is in base 2, and forcing a different radix can cause operations to be several times slower.

    One such optimisation in radix-$2$ is the \emph{reverse bit encoding}, where we can organise $X_1, \ldots, X_k$ in memory so that the entire FFT algorithm can be performed in-place (all data transformation happen inside the array the original data was passed in). This is done by putting $X_i$ at index $\tx{rev}(i, n)$, which is the function which reverses the order of the $n$-bit representation of $i$.

    One technique to avoid padding is to round to a transform size that is the product of a power of two and several small primes.
    This is so that the bulk of the computation can be done using the previous method and then when we get to a small base case we will handle the non-power of two transform lengths. This is how many FFT software libraries work.
\end{remark}


\section{Sch\"{o}nage and Strassen Integer Multiplication Algorithm}
\label{sec:schon-strass}

(TODO I have not taken into account the fact that n might not be square and we also need to round to the largest power of two)
(TODO This is their second integer multiplication scheme)

The problem with multiplying polynomials in $\Z$ is that $\Z$ does not have roots of unity. One can consider $\Z[x]$ as a subring of $\C[x]$ and then apply the FFT algorithm, rounding to the nearest integer to convert the result back into $\Z$. That is a popular approach for medium-sized inputs; however, often when we work in $\Z[x]$ we are interested in exact solutions whereas when we work in $\C$ we accept that there will be some error. If the error becomes too large then rounding to the nearest integer will yield the incorrect result, and so one must be able to guarantee the error is suitably controlled. Indeed this is how \cite{nlogn} achieves $\M{O}(n\log n)$ complexity, and it is not a simple feat. For now, we will look at the most popular algorithm for multiplication of polynomials in $\Z[x]$ with large degree sizes, which is Sch\"{o}nage and Strassen's second integer multiplication. This can be generalised to polynomials easily via Kronecker substitution or modifying this algorithm. The reason this algorithm wasn't completely The algorithm was originally formulated for integer multiplication, but can generalised for polynomials polynomials in $\Z[x]$ using Kronecker substitution from Chapter \ref{chp:preliminaries}.

\medskip


Note however that the complexity of multiplication in $\Z[x]$ is usually parameterised by the degree of the polynomial, whereas for integer multiplication it is usually parameterised by the number of bits needed to represent the integer. This highlights a fundamental difference in how we measure certain operations since the former ignores the size and cost of multiplying the integer coefficients, whereas such an assumption would trivialise the latter case. This is because the former uses the RAM computation model whereas the latter uses the Turing model. 

The Sch\"{o}nage and Strassen Integer multiplication algorithm works by reorganising the polynomial into another form where the coefficients belong to a ring that does have roots of unity, so-called \emph{synthetic roots}.

Observe that in the ring $\Z[X] / (X^n + 1)$ the polynomial $X$ is a $2n^{\tx{th}}$ root of unity. To convert our original polynomial into one with coefficients in a ring of the above form, we can use Kronecker substitution to partition the polynomial into $m = 2^{\ceil{k/2}}$ pieces with the substitution $y = x^m$. We can then apply the FFT algorithm: The FFT step takes $\M{O}(m\log m \times 2^{\floor{\frac{k}{2}}}) = \M{O}(n \log n)$ where the arithmetic on the $m$ blocks is $\M{O}(n^{\frac{1}{2}})$ time and the main algorithm is $\M{O}(n^{\frac{1}{2}}\log n^{\frac{1}{2}})$. To perform the elementwise multiplications we recursively call the algorithm again. This gives us the recursive form
\[
    C(n) < n^{\frac{1}{2}}C(n^{\frac{1}{2}}) + \M{O}(n \log n).
\]
This can be solved to obtain $\M{O}(n \log n \log \log n)$.

Modern Computer Algebra's Generalisation

Let $R$ be a ring sch that $2$ is a unit in $R$, $n = 2^k$ for some $k \in \N$, and $D = R[x] / (x^n + 1)$. In general, D is not a field. However observe that $X$ is a primitive root of unity of order $2n$. 

To multiply two polynomials $f, g \in R[x]$ with $deg(fg) < n$ it is clearly sufficient to compute $fg \in R[x] / (x^n + 1)$. This is also known as the \textit{negative wrapped convolution}. We let $m = 2^{\floor{k / 2}}$ and $t = n/m = 2^{\ceil{k/2}}$, and apply Kronecker substitution under the substitution $y = x^m$. To obtain $f^\prime, g^\prime \in R[x][y]$. Now it is sufficient to compute $f^\prime g^\prime \in R[x][y] /(y^t + 1)$ since

\[
  f^\prime g^\prime = h^\prime + q^\prime (y^t + 1) \equiv h^\prime \mod (y^t + 1)
\]
implies
\[
    fg = h^\prime (x, x^m) + q^\prime(x, x^m)(x^{tm} + 1) \equiv h^\prime(x, x^m) \mod (x^n + 1)
\]

We now take the primitive 4mth root of unity $\omega = x \in R[x] / (x^{2m} + 1)$. We wish to compute $h^\prime \in R[x, y]$ with $deg_y h^\prime < t$ satisfying the previous equations (which uniquely determines it?). Comparing coefficients of $y^j$ for $j \ge t$ we see that $\deg_x q^\prime \le \deg_x (f^\prime g^\prime) < 2m$ and conclude that 
\[
    \deg_x h^|pimre \le \max \{ \deg_x(f^\prime g^\prime), \deg_x q^\prime \} < 2m
\]
With $f^\ast = f^\prime \mod (x^{2m} + 1)$, and similarly for $g$ and $h$. We have
\[
    f^\ast g^\ast \equiv h^\ast \mod (y^t + 1) 
\]
Since the three polynomials have degrees in $x$ less than $2m$ by the previous equation, reducing them modulo $x^2m + 1$ is just taking a different algebraic meaning of the same coefficient array, in particular, the coefficients of $h^\prime \in R[x][y]$ can be read off the coefficients of $h^\ast \in D[y]$.

Computationally nothing happens in mapping $h^\prime$ to $h^\ast$, but we are no in a situation where we can apply the machinery of the FFT to compute an equation as follows. Since $t$ equals either $m$ or $2m$, $D$ contains a primitive $2t$th root of unity $\eta$
Thus one of the previous equations is then

\[
    f^\ast(\eta y)g^\ast(\eta y( \equiv h^\ast*\eta y) \mod ((\eta y)^t + 1)
\]
Given $f^\ast(\eta y)$ and $g^\ast(\eta y)$ we can use the FFT to compute $h^\ast(\eta y)$ with $O(t \log t)$ operations in $D$, using three $t$-point FFTs. A multiplication in $D$ is again a negatively wrapped convolution over $R$ which can be handled recursively.

This uses $O(n \log n \log \log n)$ operations in $R$.

% TODO get someone to check this
Integer multiplication for integers of length $n$ can be performed with $O(n \log n \log \log n)$ word operations, since we could just set $R = \Z / 31\Z$, and use Kronecker substitution to convert the integer into a polynomial in $R[x]$.
