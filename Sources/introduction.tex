\chapter{Introduction}\label{introduction}
\addcontentsline{toc}{chapter}{Introduction}


Polynomial multiplication is a fundamental problem in computational mathematics. Not only does it have numerous practical applications, but it can be generalised to a broader class of problems such as integers multiplication, calculating the Discrete Fourier Transform and evaluating convolutions. Despite being widely used, the first significant algorithmic advancement was in 1962 by Karatsuba \cite{karatsuba} who presented his discovery a week after attending a seminar by Kolmogorov who conjectured that no such improvement was possible. This development, combined with the increasing demand for efficient digital processing, quickly saw a pique of interest in the mathematical community with several subsequent multiplication algorithms being produced shortly after. Many of the algorithms developed in this 10 year period are still popular today, they included the Toom-Cook algorithm, Rader's trick, and FFT based algorithms after the FFT's rediscovery in 1965. As the computational sciences sought to understand progressively more complex systems, the was a need to multiply larger polynomials for elementary problems such as solving systems of polynomial equations, computing Gr\"{o}bner bases, and evaluating DFTs. 

% The schoolbook method of multiplication experiences a quadratic increase in execution time as the size of the inputs increase, which renders it infeasibly for most moderately sized multiplication problems. 

\medskip

The focus of this thesis is on studying and cataloguing the most significant advancements in the field, starting from the first improvement over the schoolbook method by Karatsuba, and ending with the recent improvement by Harvey and van der Hoeven \cite{nlogn}. Numerous factors affect the efficiency and validity of polynomial multiplication algorithms. The two of most interest to us are sparsity and the coefficient algebra of the polynomials. Analysis of the algorithms will be performed within two different contexts, namely their practical performance for a suitable range of inputs, and asymptotic nature as the size of the inputs grows increasingly large. The practical analysis has been supplemented with empirical tests using the nPoly\footnote{https://github.com/willcsm/nPoly} polynomial library for the Rust programming language which was developed for the purposes of this paper.

Due to the variety of such algorithms, most computer algebra systems still rely on old multiplication schemes which can work across a broad range of inputs at the cost of performance. One of our goals is to analyse the practical complexity of such methods to develop a heuristic for computer algebra systems to select the most efficient algorithm for the given inputs.

\section{Structure of Thesis}
\label{sec:Structure-of-Thesis}

To formalise our analysis, we first need to formalise the computational model of our algorithms. Since there is no ubiquitous model, Chapter \ref{chp:preliminaries} provides a mathematical refresher and summary of necessary complexity-theoretic definitions and notation. In particular, we present a brief introduction to the two computation models we will be using throughout this thesis; namely the Random Access Machine (RAM) and the Turing machine.

Chapter \ref{chp:classical} gives a brief overview of several important classical algorithms. These are the most widely implemented algorithms due to their simplicity, generality of a wide range of coefficient algebras, and efficiency for polynomials of small degrees. Popular computer algebra systems such as Macaulay2 \cite{macaulay2-polynomial}, Maxima \cite{maxima-karatsuba}, NTL \cite{ntl}, and Magma \cite{magma} all use at least one of these algorithms.

Chapter \ref{chp:eval-interp} looks at the \emph{evaluation-interpolation strategy} for multiplying polynomials. We will show how Karatsuba's algorithm from Chapter \ref{chp:classical} can be re-expressed in this framework. We will then look at applying the Cooley-Tukey Fast Fourier Transform (FFT) to multiply polynomials in $O(n \log n)$ time in the RAM model, which provides the foundations for the popular class of FFT-based algorithms. We finish with a generalisation of Sch\"{o}nage and Strassen's integer multiplication algorithm for polynomials. The FFT-based approach marks the beginning of more exotic algorithms which go beyond the coefficient-agnostic transformations in the previous chapter, and which are often only used in specialised applications.

Chapter \ref{chp:integer-rings} looks at algorithms for polynomials in integer rings. Multiplications in this ring circumvents the main shortcoming associated with algebras from previous sections, namely the increasing computational cost of ring operations in unbounded algebras (e.g. $\R$, $\C$, $\Z$). This allows us to optimise our algorithms for a better practical complexity. This area is of particular interest when computing polynomials of large degree \cite{crt-parallel-mul}\cite{crt-mul-gpu}, and other areas optimised for speed such as cryptography and signal processing. Efficiency is achieved through appropriately chosen integer rings which admit efficient Number Theoretic Transforms (NTT); a generalisation of the FFT for integer rings.

Chapter \ref{chp:asymptotic} looks at the most recent advancement in the theoretical complexity of polynomial multiplication, that is, Harvey and var der Hoeven's $O(n \log n)$ integer multiplication algorithm. Despite the algorithm \cite{nlogn} being formulated for integers, we will demonstrate how this result can be directly applied to the problem of polynomial multiplication using the techniques from Chapter \ref{chp:classical}. The theoretical bound is achieved by constructing a multi-dimensional convolution which limits the error in its finite precision arithmetic. We will also present another algorithm by Harvey and van der Hoeven that achieves the same bound but is conditional on an unproven hypothesis. It takes a more intuitive approach which naturally generalises the result for finite fields.
