\chapter{Introduction}\label{introduction}
\addcontentsline{toc}{chapter}{Introduction}


Polynomial multiplication is really good. Can use it to multiply integers and vice-versa. Is used in solving systems of polynomial equations, making Gr\"{o}bner bases, dividing polynomials etc. I recently found out it can be used in modelling cellular automata. Also can be used to evaluate DFT's through Bluestein's transform

% TODO Fill in Karatsuba date here
This thesis focuses on studying and cataloguing the most popular advancements, starting the first improvement by Karatsuba in (XXXX) all the way to the most recent paper by Harvey and van der Hoeven for multiplying integers in $O(n\log n)$ time. A polynomials package has been developed alongside this project which implements many of the algorithms for the Rust programming language.

\subsection{Goals}%
\label{sub:goals}

\begin{itemize}
    \item Provide an overview of the main classical algorithms
    \item Present practical methods that are readily available for real-world use
    \item Present some of the state-of-the-art algorithms in the area in asymptotic complexity
    \item Demonstrate through a program written in conjunction with this paper how these things can be applied
\end{itemize}


\subsection{Structure}%
\label{sub:Structure}

The structure of the document

\begin{itemize}
    \item Classical algorithms, links to integer multiplication
    \item Evaluation-Interpolation: Reformulate the classical algorithms in the Evaluation-Interpolation way and then present the DFT-based algorithm
    \item Finite fields: Present algorithms for finite fields
    \item Asymptotically superior algorithms
\end{itemize}
