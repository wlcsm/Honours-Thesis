\chapter{Implementation Details}\label{chp:implementation}

Now we will look at some additional aspects of the practical implementations of the algorithms in previous chapters. Namely we will look at sparse multiplication and practical run time analysis of our nPoly library\cite{npoly}. Sparse multiplication is used when the number of non-zero terms in the polynomial proportion to the degree of the polynomial is low. The complexity of the algorithms we have covered so far have all been solely dependent on the degree of the polynomial and so these algorithms will perform poorly for sparse polynomials. Now we will look at a technique for sparse polynomial multiplication, this is also useful as it may be used as a base case for any of the algorithms we have covered here.

For example if we were to use Karatsuba's algorithm on $1 + x^2 + x^{60} + x^{100}$, then the algorithm would need to recurse $8$ times to reach a base case of one. However, if we organise our polynomial into a sparse vector and set a base case at $\# f \le 2$, then we only need to recurse once. Thus by formulating our base case in terms of the number of non-zero term in conduction with this algorithm, we may achieve a nice speedup. This algorithm also has the advantage for working equally over multivariate polynomials as well, which tend to be sparse in practice.

Note that this doesn't change the current complexity analysis from previous chapters as they were formulated to be independent of whether the coefficients were zero or non-zero, thus they used coefficient vector representations.

\section{Sparse Polynomial multiplication}

Previously in Chapter \ref{chp:classical} we showed that the schoolbook multiplication method has complexity $\M{O}(nm)$ for input polynomials of degree $n$ and $m$. However if our polynomials are reasonably sparse, it is natural to ask if we can design and algorithm with a complexity that is only dependent on the number of non-zero terms in the polynomial, rather than the maximum degree.

It would seem reasonable to try to use the schoolbook method to achieve this, however a problem arises when coordinating the access of elements in memory. Once we have multiplied two terms together, we need to include this result into the collection of terms we have calculated so far.

There are several methods we might look using to achieve this:
\begin{itemize}
    \item a coefficient vector. This requires potentially $\M{O}(n)$ time to shift data to allow the new term to be inserted,
    \item a binary tree would provide $\log n$ insertion times, but uses expensive memory operations and is not memory efficient, 
    \item a hashmap would provide $\M{O}(1)$ insertion time, but incurs additional costs when hashing the elements. Furthermore, since element in a hashmap are not store in any particular order, the elements would need to be sorted in the end, and so the complexity would be $m \log m$ where $m$ is the number of elements in the hashmap. This is more suitable in the case of multiplying many polynomials together as we then only need to sort the array once at the end.
\end{itemize}

Let $\# g$ denote the number of non-zero terms in the polynomial $g$. Say we are part way through multiplying two polynomials $a, b \in K[x]$. Denote the cumulative result after $k$ steps as $c_k$ e.g. after $\# a \# b$ steps we will have calculated the entire result so $c_{\#f + \#g} = ab$.

Using the formula
\[
    ab = \sum^{n + m}_{k=0} \bb{\sum_{i + j = k}a_ib_j} x^{i + j}.
\]
When we calculate another term $a_ib_j x^{i + j}$ for some $1 \le i \le n$ and $1 \le j \le m$, we now need add this to the resulting polynomial we have calculated so far. To find the matching monomial in our result polynomial (or insert it if it doesn't exist) we will need to perform a binary search. However, if we have stored the result as an array, then inserting the monomial could cause an $\#c_k$ shift in elements. If we store is as a linked list then this is not very efficient due to the expensive memory read operations.

Here I will present a technique for multiplying polynomials in $O(\# f \# g \log (\min\{\# f, \# g\})$. The main idea is to compute the monomials in order (as much as possible), so the output is naturally sorted. Later research found that this technique is similar to Johnson's sparse multiplication algorithm \cite{johnson-sparse-polynomial}.

As a practical example, we know that the smallest terms will be the lowest term of $f$ multiplied by the lowest term of $g$. Then the next smallest term in the result is either the smallest term of $f$ multiplied by the second smallest term of $g$, or vice versa. So we must compare them and output the smallest.

In general, fix a monomial order, then let $x_{ij}$ be the $i^{\text{th}}$ smallest monomial of $f$ multiplied by the $j^{\tx{th}}$ smallest monomial of $g$. Then $x_{ij} < x_{kj}$ when $k > i$ and $x_{ij} < x_{ik}$ when $k > j$. Thus before performing any calculations, we are able to deduce a partial ordering on the collection $\{x_{ij}\}_{i, j=0}^{i=\# f, j = \#g}$. The goal of the algorithm is to then resolve this into a total ordering in an efficient manner.

\begin{figure}
    \center
    \begin{tikzpicture}[baseline= (a).base]
        \node[scale=.8] (a) at (0,0){
                \begin{tikzcd}[column sep=small]
         & & & x_{\# f \# g} \arrow[ld] \arrow[rd] & & & \\
         & & x_{\# f (\# g - 1)} \arrow[ld] \arrow[rd] & & x_{(\# f - 1)\#g} \arrow[rd] \arrow[ld] & & \\
         & \cdots \arrow[ld] \arrow[rd] & & x_{(\# f - 1)(\# g - 1)} \arrow[rd] \arrow[ld] & & \cdots \arrow[rd] \arrow[ld] & \\
                    x_{\# f 1} \arrow[rd] & & \cdots \arrow[ld] \arrow[rd] & & \cdots \arrow[ld] \arrow[rd] & & x_{1 \# g} \arrow[ld] \\
                                          & \cdots \arrow[rd] & & x_{22} \arrow[rd] \arrow[ld] & & \cdots \arrow[ld] & \\
                                          & & x_{21} \arrow[rd] & & x_{12} \arrow[ld] & & \\
                                          & & & x_{11} & & &
                \end{tikzcd}
            };
    \end{tikzpicture}
    \caption{The partial order that is known a priori}
\end{figure}

The idea is that we start with $x_{ij}$ then we compute $x_{(i+1)j}$ and $x_{i(j+1)}$ and add them to the heap. We then take out the smallest and add it to the resulting vector. This is like starting from the $x_{11}$ and going up in the graph. In there is a path of arrows from one node to another, then that implies that node is definitely greater than it. It can then be seen from this graph that the most number of node that you can have in the heap at any time, occurs as a straight horizontal line across, which is $\min\{\# f, \# g\}$.


Therefore the multiplication operation will be $O(\# f \# g \min\{\# f, \# g\})$.


% \subsection{Sparse polynomial (my way)}

% The idea is that we compress the polynomial (i.e. ignoring the zero terms), then do a fft in that compressed state, multiply, then expand back into full
% e.g. Imagine $(x^{20} - 1)(x^{40} - 1)$, then I can replace the exponents with $n = 20$ and $m = 40$ and calculate $(x^n - 1)(x^m -1)$, which gives me $x^{n+m} - x^n - x^m + 1$, then I could have actually chosen $n = 1$ and $m = 2$, and calculated $(x - 1)(x^2 - 1)$ via the fft and gotten $x^3 - x^2 - x + 1$ and then expanded back into $x^{60} - x^{40} - x^{20} + 1$. Thats the basic premise

% I like the hybrid approach in the Fast Poly Mult paper. Also the mixed basis or small prime one

\section{Runtimes}

nPoly is a library for polynomial arithmetic written in the Rust programming language. Rust is a modern systems programming language that has become increasingly popular in recent years for its ability to make highly-performant programs with clean code.
One of the advantages of Rust is its support for generic programming which has enabled nPoly to implement its algorithms over all algebras the algorithms support. Currently, nPoly supports: polynomial addition/subtraction/evaluation, Schoolbook multiplication, Karatsuba multiplication, FFT multiplication, Kronecker substitution and limited support for Gr\"{o}bner bases.

\begin{center}
    \begin{tabular}{|c| c c|}
        \# El. & FFT & STD\\
        8      & 191950 & 106140 \\
        16     & 277790 & 289350 \\
        64     & 934627 & 1859690 \\
        256    & 4282801 & 23703833 \\
        1024   & 17969459 & 264139561 \\
        2048   & 33150304 & 188562822 \\
        4096   & 78430218 & 4340959519 \\
        8192   & 126315317 & 16,154233194 \\
    \end{tabular}
\end{center}

% \subsection{Sparse evaluation}

% So i've got an algorithm here for some sparse multiplication technique.

% The basic premise is that we consider the tree created by the FFT and then we want to cut off branches, but by ``cut off'' I mean, ``never actually create in the first place''. So to do that we do as follows

% Construct the tree below.
% Then we add the elements in their reverse bit order. When we add an element, we slide it down the tree as far right as possible. All the ones that slide to the left need to have a calculation first.

% Note: We can probably leverage the intermediate results from the bucket sort, for instance, if one bucket is empty after the first pass, then we know that that subtree is actually empty. So if the bucket sort is inefficient, then actually that implies that your underlying data in not random and can actually be leveraged because you know whole subtrees will be empty

% \begin{algorithm}[H]
%     \SetAlgoLined
%     \KwData{Vector of monomials to be transformed (non-expanded)}
%     \KwResult{Fourier transform of the input coefficients}
%     rbe $\gets$ Reverse Bit Encoding of monomials\;
%     acc $\gets$ Empty vector\;
%     \For{el in rbe}{
%         acc.push(el)\;
%         \While{acc.len() > 1 \&\& canCombine(acc[-2], acc[-1])}{
%             tmp $\gets$ acc.pop()\;
%             acc[-1].combine(tmp)\;
%         }
%     }
%     \Return{acc}
%     \caption{Sparse FFT}
% \end{algorithm}


% \begin{algorithm}[H]
%     \SetAlgoLined
%     \KwData{arg1, arg2: Two subtrees}
%     \KwResult{Boolean}
%     depth $\gets$ findLowestConnection(arg1.path, arg2.path, arg1.depth, arg2.depth)\;
%     \Comment{Note that simply constructing such a number of all 1's and testing if equal would be quicker}\;
%     \For{i in 0 ... (node - arg2.index)}{
%         \If{arg2.index \& 1 $<<<$ i = 0}{
%             \Return false
%         }
%     }
%     \Return{true}
%     \caption{canCombine}
% \end{algorithm}

% \begin{algorithm}[H]
%     \SetAlgoLined
%     \KwData{path1, path2: Two subtrees; depth1, depth2: Usize}
%     \KwResult{Absolute depth of the connecting tree: usize}
%     \eIf{depth1 > depth2}{
%         path2 >>= depth1 - depth2\;
%         depth2 = depth1
%         }{
%         path1 >>= depth2 - depth1\;
%         depth1 = depth2
%     }
%     count $\gets$ 0\;
%     \While{path1 $\neq$ path2}{
%         count += 1\;
%         path1 >>= 1\;
%         path2 >>= 1\;
%     }
%     \Return{count + depth}
%     \caption{findLowestConnection}
% \end{algorithm}

% \begin{enumerate}[1.]
%     \item Organise the list into its Reverse Bit encoding ($O(t \log t \log n)$). The $\log n$ term arises from the degrees of the polynomials are literally have $\log n$ space complexity
%     \item Add a new element to the vector and see if it can be combined with the previous element or we need to wait to see if it needs to be combined with the next element first
%     \item To combine, we ``expand'' the two elements into the appropriate size and then do the combination step from the FFT
%     \item Continue until all the elements have gone
%     \item Finish by combining all the elements remaining in the list into the final transform
% \end{enumerate}

% Unfortunately in this one we still end up expanding the polynomials in the end so it must be $O(n\log n)$. But we should try not to do that.\\
% The best result would be if I could interpolate the results whilst still in their $O(t)$ space complexity format. Then we would also have a good representation for our resulting polynomial if we wanted to do calculations based on that.

% \section{Search trees for membership in monomial ideals}

% \textbf{Problem:} Given a monomial ideal, want to test to see if a monomial is in it.

% The time to beat is $O(nm)$ where $n$ is the number of generators in the ideal and $m$ is the number of indeterminates. This is obtained by a linear search

% The next best thing would be to try and do a search based off each of the values in the multi-index. You would try to do a binary search on th first value, then on the second, and so on. The problem is that it is pretty hard to see what kind of data structure this might use.

% Another optimisation would be to record the highest index value for each of the indices. Then we might be able to fit multiple indices inside a single machine word.

% The division relation "|", where $a | b$ for $a$ and $b$ monomials. Is independent of monomial orderings, and is reflexive and transitive. But it is not a total order so we cannot perform a normal binary search. In fact, if we have that one generator of the ideal divides the other, then the other can be eliminated so it is unclear how we would even go about searching for divisibility.

% What we can do is:
% \begin{enumerate}
%     \item Order then by total degree, if $\tx{totdeg}(a) < \tx{totdeg}(b)$ then certainly $b \not | a$. This can give us a rough estimate of where to place it
%     \item Order by total degree of the first half of the multi-index. Say if we have a monomial with multi-index $(1, 0, 5, 6, 2, 4)$, then we have its total degree is $1 + 5 + 6 + 2 + 4 = 18$, then the total degree of the first half is $1 + 0 + 5 = 6$.
%     \item Then order by their second half and so forth. This way we can build up a key for the monomial.
% \end{enumerate}

% This still has some of the same problems as the previous method of just searching through the indices but I think on average it will be quicker because the variance in the variables should be lower when you sum them and so they should be more accurate.

% Actually I would propose this as a new monomial order. It might allow you to obtain tighter bounds on certain algorithms.
